\begin{figure}[H]
    \centering    \includegraphics[width=0.8\textwidth]{Figures}
    \caption{This figure...}
    \label{fig:}
\end{figure}







\subsection{The measuring devices}
\subsubsection{The meteorological measuring devices}
The wind data from both Hörby and Helsingborg used the high-performance wind sensor Vaisala WAA15A for the wind speed and Vaisala WAV15A for the wind direction. These instruments were serviced and calibrated every year or every other year, and had been in use since 1995. The WAA15A anemometer measured wind speed with an accuracy of $\pm\SI{0.17}{\m\per\s}$, and the WAV15A wind vane measured the wind direction with an accuracy better than $\pm\ang{3}$ \cite{vaisalaWindSetWA152021}. The WAA15A anemometer works by a rotating chopper disc that interrupts an infrared beam, resulting in a laser pulse proportional to the wind speed. The WAV15A wind vane uses a counterbalanced vane with an optical disc. When the vane turns, infrared LEDs detect the change in angle with the disc and phototransistors, resulting in a precise measurement of the wind angle. For rain monitoring, the Geonor T200 device had been in use for all stations since 1995. Like the wind monitor, this device had been serviced and calibrated every year or every other year. This device works by measuring precipitation with a vibrating wire sensor that detects weight changes from the water droplets\cite{geonorinc.T200BSeriesAll2019}. The device has a measurement accuracy better than \SI{0.1}{\mm}. The rain measurements in Tånga was measured manually by a beaker located on a field. Thus, errors in this measurement were higher than the digital measurement devices . 

The barometer that has been in use for Helsingborg is a Vaisala PTB201A for the entire period, except for the periods from April 15, 2015, to April 17, 2025, and from September 19, 2004, to May 23, 2014, when a Vaisala PTB220 was used instead. Even then, the device has been serviced every year or every other year. The PTB201A digital barometer operates using a silicon capacitive absolute pressure sensor, providing stable and accurate pressure values \cite{vaisalaPTB200DIGITAL1993}. The sensor functions by means of a flexed diaphragm inside a capacitor that bends in response to air pressure, causing a change in the capacitor’s distance and thus a variation in the current. This device measures pressure in the range of \SI{600}{\hecto\pascal} to \SI{1100}{\hecto\pascal}, with an accuracy of $\pm\SI{0.3}{\hecto\pascal}$. Errors in the device may arise due to environmental factors, such as exposure to condensing gases. The Vaisala PTB220 digital barometer operates in a similar manner but offers a wider measurement range of \SI{500}{\hecto\pascal} to \SI{1100}{\hecto\pascal}, with an improved accuracy of $\pm\SI{0.15}{\hecto\pascal}$ \cite{vaisalaPTB220SeriesDigital2001}.


- How was pressure in Ängelholm  and rain in Ängelholm measured? 

\subsubsection{The \texorpdfstring{\PM }{PM2.5} measuring devices}
The measurement device used at Vavihill was the ambient particulate monitor TEOM 1400. This monitor continuously collects airborne particles less then \SI{2.5}{\micro\g} onto a filter and measures their mass using an oscillating microbalance technique \cite{thermofisherscientificinc.TEOMSeries1400a2007}. The oscillating microbalance works by vibrating at a natural frequency, which changes as particles accumulate on the filter. Since this frequency shift is proportional to the mass of the deposited particles, their total mass can be accurately determined. The precision of the monitor was $\pm\SI{1.5}{\micro\gram\per\meter\cubed}$. Thus, the monitor provides high precision and stable measurements. However, a key limitation is that it cannot distinguish between different types of particles, as it only measures total mass.

When the measuring station was moved to Hallahus, the measuring device was updated to the fine dust analysis system Palas FIDAS 200. This device works by an optical aerosol spectrometer which samples particles from an isokinetic inlet through a polychromatic light-scattering channel where the scattering angles and intensity were measured. \cite{palasgmbhOperatingManualFidas}. This results in a high accuracy of $\pm\SI{0.1}{\micro\gram\per\meter\cubed}$, indicating an improvement from the other device. 

The monitoring station in Malmö used several measuring devices over time, in conjunction with other equipment. Between the start of the monitoring and January 1, 2009, the TEOM 1400 monitor was used. From January 1, 2009, to December 31, 2015, the TEOM 1400, FDMS, and 8500 B or CB dryer were employed. Between January 1, 2016, and December 31, 2021, the TEOM 1405F and FDMS systems, along with the 8500 B, were in use. Finally, from January 1, 2022, the Palas FIDAS 200 monitor replaced the earlier systems. The FDMS (Filter Dynamics Measurement System) is a dynamic filter measurement system that enhances measurements by accounting for volatile and semi-volatile particles \cite{thermoscientific8500FDMSFilter2010}. The CB dryer and 8500 B were air dryers used to prevent moisture from entering the measurement devices, ensuring accurate data by avoiding potential interference caused by water vapour.

