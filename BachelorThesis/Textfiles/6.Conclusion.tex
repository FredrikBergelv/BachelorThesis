\section{Conclusion}
This thesis has demonstrated that the presence of of high-pressure blocking events in southern Sweden increased the aerosol concentration significantly in both rural as urban locations. Increased levels of \PM was seen in both location after nine to 13 days of uninterrupted high-pressure blocking event, indicating accumulation of aerosols during the event. This accumulation was mainly attributed to the presence of a subsidence inversion layer, which prohibited vertical air mixture. An stronger increase was found in the urban area of Malmö than the rural area of Vavihill, which was mainly attributed to local emissions in the urban location. The wind directional dependence on \PM indicated that the long-range transport of aerosols is a significant contributor to aerosol concentration in the region, where winds from central Europe significantly increased the aerosol levels whereas winds from the northeast did not effect the concentrations. The seasonal dependence indicated an increase during the winter, which supports the idea of local emissions affecting the aerosol concentration. Observing the lengths and number of high-pressure blocking events showed that there has not been an increase of any sort during the last 74 years. 
