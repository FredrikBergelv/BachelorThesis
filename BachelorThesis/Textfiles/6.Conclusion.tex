\section{Conclusion}
This thesis has demonstrated that the presence of high-pressure blocking events in southern Sweden significantly increased the aerosol concentration in the rural and urban location. Increased levels of \PM were seen in both locations after 9 to 13 days of uninterrupted high-pressure blocking events, indicating accumulation of aerosols during the event. This accumulation was mainly attributed to the presence of a subsidence inversion layer, which prohibited vertical air mixing. A stronger increase was found in the urban area of Malmö than the rural area of Vavihill, which was mainly attributed to local emissions. The wind directional dependence on \PM concentrations indicated that the long-range advection of aerosols was a significant contributor to aerosol concentration in the region, where winds from central Europe significantly increased the aerosol levels, whereas winds from the northeast did not affect the concentrations. The seasonal dependence indicated an increase during the winter, which supports the idea of local emissions affecting the aerosol concentration. Observing the lengths and number of high-pressure blocking events showed that there has not been an increase of any sort during the last 74 years.

