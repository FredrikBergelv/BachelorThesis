\newpage
\section{Conclusion and Summary}
This thesis has demonstrated that the presence of high-pressure blocking events in southern Sweden significantly increased the aerosol concentration in both rural and urban locations. The mean \PM in Malmö went from \SI{10}{\micro\gram\per\meter\cubed} to a maximum of \SI{21}{\micro\gram\per\meter\cubed} during the high-pressure blocking event, and an increase from \SI{7}{\micro\gram\per\meter\cubed} to a maximum of \SI{18}{\micro\gram\per\meter\cubed} could be observed in Vavihill during the event. Increased levels of \PM were seen in both locations after eight to thirteen days of uninterrupted high-pressure blocking, indicating accumulation of aerosols during the events. This accumulation was mainly attributed to the presence of a subsidence inversion layer, which prohibits vertical air mixing. A stronger increase was found in the urban area of Malmö than the rural area of Vavihill, which was mainly attributed to local emissions and a slightly stronger ground inversion, especially during the winter. The wind directional dependence on \PM concentrations indicated that the long-range advection of aerosols was a significant contributor to aerosol concentration in the region, where winds from central and eastern Europe significantly increased the aerosol levels. The seasonal dependence indicated an increase during the winter. A stronger increase was also observed with stronger high-pressure blocking events, where the highest levels of \PM could be observed. Observing the lengths and number of high-pressure blocking events showed that there has not been an increase of any sort during the last 74 years, which aligns with other European studies.

To summarize, the analysis of aerosol concentrations during meteorological events is important for monitoring the health risks associated with different meteorological phenomena. This thesis has shown that not only domestic emissions play a role in air quality, but that advection from nearby countries also plays a crucial role. This shows the magnitude of international work to prohibit the large-scale emissions of aerosols. As the climate changes, so does the weather, and understanding how these changes affect us is of utmost importance. 

