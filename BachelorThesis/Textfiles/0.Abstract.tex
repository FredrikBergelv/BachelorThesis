\text{}
\vspace{2cm}
\begin{abstract}

    \noindent High-pressure blocking events are anticyclones that persist over a region for a long time. The event prevents vertical mixing of aerosols due to the subsidence inversion created by the adiabatic compression of air during the anticyclone. This study investigated the relationship of aerosol concentrations during periods of high-pressure blocking events in southern Sweden for the rural location of Vavihill and the urban location of Malmö. A total of 143 and 222 high-pressure blocking events were found in Vavihill and Malmö respectively between 1995 and 2024. The requirements for a high-pressure blocking event was: A pressure above \SI{1014}{\hecto\pascal}, the event lasting for at least 5 days, and has a maximum amount of rainfall of \SI{0.5}{\mm\per\hour}. The high-pressure blocking event was compared to mean \PM concentrations from the same locations for each hour since the beginning of the event. The data was also sorted into wind direction, season, and pressure strength for further analysis. Using Mann-Kendall statistics and standard deviations, the data showed a significant increase of \PM for both locations during periods of high-pressure blocking, especially in Malmö where local emissions and a stronger subsidence inversion provided a significant increase. Peak aerosol levels were achieved after 9 to 13 days, indicating an accumulation of aerosols during the event. The wind dependence showed a stronger increase with winds from the southeast, indicating advective transport of aerosols to the region, although local emissions also played a significant role. This was supported by the findings of the seasonal and pressure strength dependencies. Although no increase of high-pressure blocking events were found in the region, the impact on air quality from high-pressure blocking events is still crucial. These results show the dependence of aerosol concentrations on atmospheric events.
    \\
    \text{}
    \\
    \textbf{Keywords:} Aerosols, \PM, high-pressure blocking, Mann-Kendall, southern Sweden

\end{abstract}


\vspace*{\fill}
Cover photo taken by SeaWiFS Project, NASA, on March 28 2003 \cite{nasaHazeEurope2003}. 