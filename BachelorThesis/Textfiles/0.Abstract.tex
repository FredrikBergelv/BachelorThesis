\text{}
\vspace{2cm}
\begin{abstract}

    \noindent High-pressure blocking events are anticyclones that persist over a region for a long time. The event prevents vertical mixing of air pollutants due to the subsidence inversion created by the adiabatic compression of air during the anticyclone. This study investigated the relationship of aerosol concentrations during periods of high-pressure blocking events in southern Sweden for the rural location of Vavihill and the urban location of Malmö. A total of 247 high-pressure blocking events were found in Vavihill and Malmö between 1995 and 2024. The requirements for a high-pressure blocking event were: A pressure above \SI{1014}{\hecto\pascal}, the event lasting for at least 5 days, and having a maximum amount of rainfall of \SI{0.5}{\mm\per\hour}. These criteria had to be fulfilled for the initial five days in both locations with a maximum time difference of \SI{5}{\hour}. The \PM levels during the progression of the high-pressure blocking events were compared to mean \PM concentrations from the same location. The data was also sorted according to wind direction, season, and pressure strength for further analysis. Using Mann-Kendall statistics and standard deviations, the data showed a significant increase of \PM for both locations during periods of high-pressure blocking. Higher \PM concentrations were observed in Malmö due to larger local emissions, and a slightly stronger ground inversion. Peak aerosol levels were achieved after 8 to 13 days, indicating an accumulation of aerosols during the event in both locations. The investigation showed a stronger increase with winds from the southeast, indicating advective transport of aerosols to the region, although local emissions also played a significant role. This was supported by the findings of the seasonal and pressure strength dependencies, where stronger high-pressure blocking events showed the highest levels of \PM. Although no increase in the frequency of high-pressure blocking events was found in the region, the impact on air quality from high-pressure blocking events is still crucial. These results show how aerosol concentrations depend on atmospheric events in the region.
    \\
    \text{}
    \\
    \textbf{Keywords:} Aerosols, \PM, high-pressure blocking, Mann-Kendall, southern Sweden, Skåne

\end{abstract}



\vspace*{\fill}
Cover photo taken by SeaWiFS Project, NASA, on March 28 2003 \cite{nasaHazeEurope2003}. 