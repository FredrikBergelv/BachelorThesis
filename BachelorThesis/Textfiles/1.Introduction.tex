\section{Introduction}
It is common knowledge that Earth's increasing temperature has many side effects. One such effect is the increase in frequency of extreme weather phenomena \cite{mitchellExtremeEventsDue2006}. One such phenomenon, which lacks extensive research, is high-pressure blocking events. A high-pressure blocking event is an anticyclone that covers an area for a prolonged period of time and often blocks other types of weather, hence the name. This results in limited cloud formations and a pronounced daily temperature variation \cite{lupoAtmosphericBlockingEvents2020}. However, an anticyclone is also associated with slower air movement and wind, causing the air to remain stagnant. This can lead to an accumulation of air pollutants such as aerosols \cite{caiImpactBlockingStructure2020}.

To investigate the relationship between aerosols and high-pressure blocking events, one must analyse periods of high-pressure blocking and examine the concentration of aerosols during these periods. Thus, the aim of this thesis is to: identify a suitable method for detecting periods of high-pressure blocking using pressure data from the \textit{Swedish meteorological and hydrological institute} (SMHI); analyse these periods in relation to \PM levels from rural (Vavihill, Svalöv, Skåne County) and urban (Malmö, Skåne County) areas. Additional relevant data, such as wind direction, season, and pressure strength will be examined to gain a comprehensive understanding of high-pressure blocking events and their characteristics. This thesis will also explore the frequency of high-pressure blocking events to determine whether this weather phenomenon is becoming more common, which is particularly important if a positive correlation with aerosol levels is found. 
