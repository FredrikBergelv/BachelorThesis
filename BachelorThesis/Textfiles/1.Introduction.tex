\section{Introduction}
It is common knowledge that Earth's increasing temperature has many side effects. One such effect is the increase in frequency of extreme weather phenomena \cite{mitchellExtremeEventsDue2006}. One such phenomenon, which lacks extensive research, is high-pressure blocking. High-pressure blocking is an anticyclone that covers an area for a prolonged period of time and often blocks other types of weather, hence the name. This results in clearer weather and more extreme temperatures \cite{lupoAtmosphericBlockingEvents2020}. However, an anticyclone is also associated with lower air movement and wind, causing the air to remain stagnant. This can lead to an accumulation of aerosols such as PM\textsubscript{2.5} in the region \cite{caiImpactBlockingStructure2020}.

To investigate the relationship between PM\textsubscript{2.5} and high-pressure blocking, one must analyze periods of high-pressure blocking and examine the concentration of PM\textsubscript{2.5} during these periods. The goal of this thesis is to analyze the concentration of PM\textsubscript{2.5} during periods of high-pressure blocking by examining data from the Swedish Meteorological and Hydrological Institute (SMHI) and PM\textsubscript{2.5} data from rural (Vavihill, Svalöv Skåne län) and urban (Malmö, Skåne län) areas.
