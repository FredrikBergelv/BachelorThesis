\section{Theory}
\subsection{The physics behind anticyclones}
Anticyclones are meteorological high-pressure systems in which air sinks toward the ground, creating high pressure \cite{spiridonovCyclonesAnticyclonesSpringerLink2020}. This occurs due to the convergence of air from all directions, which forces the air to move downward. The descending air undergoes adiabatic compression, resulting in an increase in the energy of air molecules, or, in other words, a higher temperature. This rise in energy inhibits cloud formation, as the air molecules are unable to ascend due to the lack of cooling. The absence of clouds allows solar radiation to significantly impact the temperature during an anticyclone. Consequently, this leads to a large temperature difference between day and night, with summer anticyclones associated with high temperatures and winter anticyclones with low temperatures. Due to the Coriolis effect, anticyclones rotate in a clockwise direction in the Northern Hemisphere.

more on anticyclones

A high-pressure blocking period refers to a prolonged anticyclone characterized by higher surface pressure covering a large area \cite{lupoAtmosphericBlockingEvents2020}. Since the blocking system extends over a vast region, the pressure gradient remains small due to minimal fluctuations. As a result, winds tend to be calm to gentle breezes. A blocking period is typically defined as lasting between five and ten days, although no single definition exists. While the concept has been recognized in meteorology for over a century, the long-term consequences of blocking events are not yet fully understood. High-pressure blocking periods are more common in the Northern Hemisphere compared to the Southern Hemisphere. Research has indicated that the frequency of blocking periods has increased in recent years \cite{lupoAtmosphericBlockingEvents2020}. 

Recurring anticyclones can be classified into Hess and Brezowsky (1977) macrocirculation types, such as the Fennoscandian High (Hfa), the Southeast Anticyclone (Sea), and the Central European High (HM) \cite{bartholyEuropeanCycloneTrack2006}. These anticyclones are located at specific geographic points. Since anticyclones exhibit winds rotating clockwise around their center, the winds from (Hfa), (Sea), and (HM) tend to blow toward southern Sweden from the south and east. The transport of airborne pollutants, such as ozone, can occur via these winds \cite{oteroImpactAtmosphericBlocking2022}. Consequently, it can be hypothesized that other airborne aerosols, such as PM\textsubscript{2.5}, should also be transported through these wind patterns.

Moving anticyclones

\subsection{The physics behind PM 2.5}
PM\textsubscript{2.5} refers to particulate matter with a diameter of \SI{2.5}{\micro\meter} or less. Although these aerosols can form naturally in the atmosphere, their primary sources include solid fuel combustion for domestic heating, industrial activities, and road transportation \cite{europeanenvironmentagencyEuropesAirQuality2024}. A significant contributor to PM\textsubscript{2.5} pollution is the bonding of aerosols to ammonia emitted from agricultural activities. The European Union has set an annual mean limit for PM\textsubscript{2.5} concentrations at \SI{25}{\micro\gram\per\cubic\meter}. This threshold has been exceeded in several countries, including Croatia, Bosnia and Herzegovina, Italy, Poland, North Macedonia, and Türkiye \cite{EuropesAirQuality2024}. Studies have demonstrated a correlation between elevated PM\textsubscript{2.5} concentrations and an increased risk of respiratory, cardiovascular, and cerebrovascular diseases, as well as diabetes.

Since PM\textsubscript{2.5} emissions are particularly high in countries such as Poland, anticyclonic winds from (Hfa), (Sea), and (HM) are expected to increase PM\textsubscript{2.5} concentrations in southern Sweden \cite{europeanenvironmentagencyEuropesAirQuality2024}. These aerosols would be transported to southern Sweden via southerly to easterly winds during the anticyclone. If this occurs during a high-pressure blocking event, the aerosols may accumulate over the region while continuously being advected by southerly and easterly winds. Studies in China have shown that the dispersion of aerosols during high-pressure blocking is inhibited \cite{caiImpactBlockingStructure2020}. Whether the same occurs in southern Sweden is of interest for further study.

\subsection{The Mann-Kendall test}
The Mann-Kendall test is a common statistical tool in weather physics to evaluate the monotonic increase of a data series.