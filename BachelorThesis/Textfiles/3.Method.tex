\newpage
\section{Method}
The relevant data was downloaded from the SMHI’s website as CSV files. These files included hourly atmospheric pressure data, hourly rain data, and hourly wind data (speed and direction). Hourly PM\textsubscript{2.5} data, measured over one-hour intervals, was also downloaded. Since the goal of this project is to analyze PM\textsubscript{2.5} concentrations in Southern Sweden during high-pressure blocking events, one urban and one rural site were selected. The locations were chosen based on their classification as rural or urban and the length of time the stations had been in operation, where more data was considered better. 

The rural measuring station with the most data was Vavihill (Svalöv, Skåne County, Sweden). This station was active from September 28, 1999, to November 15, 2017. However, due to missing data on some days, only \SI{57}{\%} of the period contained non-NaN values. Additionally, between 2017 and 2018, the Vavihill station was relocated to nearby Hallahus, where it operated from May 10, 2018, to December 31, 2022, with \SI{93}{\%} of the period containing non-NaN values. Combining these datasets resulted in a total of 5,371 days of hourly data. For urban locations in Southern Sweden, Malmö had the most data, with measurements recorded from June 3, 1999, to December 31, 2023. Here, \SI{90}{\%} of the recorded values were non-NaN, resulting in 8,074 days of data.

The choice of atmospheric pressure measurement station was Helsingborg, located \SI{25}{\km} from Vavihill and \SI{49}{\km} from Malmö. This location was chosen based on its proximity to both PM\textsubscript{2.5} measuring stations, the fact that neither Malmö nor Vavihill have pressure measurements from this period, and the fact that the station has been in use from August 2, 1995, to October 10, 2024, with measurements taken every hour without any missing (NaN) values. This station thus covers the entire period of the PM\textsubscript{2.5} data. It is important to note that the measurements are shown as sea-level pressure.

The relevant rain and wind data were gathered from two different stations. For Vavihill, the station at Hörby, located \SI{35}{\km} away, was used, and for Malmö, a weather station just \SI{6}{\km} away was used. The weather station at Hörby was chosen instead of Helsingborg since neither Vavihill nor Hörby are located along the coast, whereas Helsingborg is. The wind and rain data from Hörby were measured from August 1, 1995, to October 1, 2020, with measurements taken every hour without any missing (NaN) values. It is important to note that the Hörby station was temporarily relocated for a short period in 2021. The rain data from Malmö was measured from November 21, 1995, and the wind data from January 1, 1990, with both measurements ending on December 31, 2020. These stations did not lack any data.

To evaluate when there was a high-pressure blocking event for Vavihill or Malmö, the rain data and atmospheric air pressure were used. For a period to be defined as a high-pressure event, the atmospheric pressure had to be over \SI{1014}{\hecto\pascal}, and the rainfall had to be less than \SI{0.5}{\mm\per\hour}. These values were based on the fact that \SI{1014}{\hecto\pascal} was the mean atmospheric pressure from Helsingborg, and \SI{0.5}{\mm\per\hour} was chosen since this is considered light rain. For a high-pressure event to be considered a high-pressure blocking event, the criteria for a high-pressure event had to persist for at least \SI{120}{\hour} (5 days). This value was chosen since a 5-day limit is often considered when classifying high-pressure blocking events.

The data was analysed by taking the mean and standard deviation of each hour of the high-pressure blocking event, from hour one, to evaluate the average progression of a blocking over time using the Python packages \texttt{NumPy} and \texttt{pandas}. Since the blocking events varied in length, the number of data points was also plotted, with the requirement that each hour consist of at least 8 data points. This resulted in a plot of the mean concentration of PM\textsubscript{2.5} for every hour from the beginning of the blocking period, for both Vavihill and Malmö. All of the PM\textsubscript{2.5} plots were evaluated together with the mean value when there was no blocking and the EU annual mean limit for PM\textsubscript{2.5}.

Afterwards, the data was sorted in different ways to explore how the PM\textsubscript{2.5} concentration depended on different parameters. Firstly, the data was sorted into one of four wind categories: North-East (310° to 70°), South-East (70° to 190°), West (190° to 310°), and no specific direction. This was done by categorizing the data if 60\% of the wind directional data fell into one of these categories, with no wind being handled as a NaN value.

Secondly, the data was sorted based on the season of the blocking. This was evaluated by taking the midpoint date of the blocking and categorizing the season by the month it occupied. December, January, and February were considered winter; June, July, and August were considered summer; and spring and autumn were the remaining months. Lastly, the data was categorized based on the strength of the high-pressure blocking, where a weak high-pressure blocking event had a mean atmospheric pressure between \SI{1014}{\hecto\pascal} and \SI{1020}{\hecto\pascal}, a medium-strength blocking event had a mean atmospheric pressure between \SI{1020}{\hecto\pascal} and \SI{1025}{\hecto\pascal}, and a strong blocking event had a mean atmospheric pressure over \SI{1025}{\hecto\pascal}.

To evaluate if the plots produced significant results, two tests were performed. The first test compared the mean and standard deviation of PM\textsubscript{2.5} during high-pressure blocking events with the mean and standard deviation during periods without high-pressure blocking. Secondly, the Mann-Kendall test was performed to evaluate whether the PM\textsubscript{2.5} mean during high-pressure blocking events had actually increased, and if so, by how much. This was done by assessing whether the p-value was below 0.05 and whether $\tau$ was above 0.5.

The second task was to evaluate whether high-pressure blocking events had become increasingly more common. This was evaluated in two different ways: by calculating the number of high-pressure blocking events per year, and the number and lengths of high-pressure blocking events per year. The number of days of blocking was also sorted by the season of the blocking to provide more insight into the nature of the high-pressure blocking events. For this evaluation, atmospheric pressure data from Ängelholm was used instead of Helsingborg due to the pressure data from Ängelholm being active from January 5, 1946, to October 1, 2024, meaning it has been in service for 49 years longer than that from Helsingborg. This station is located \SI{44}{\km} from Vavihill and \SI{76}{\km} from Malmö. However, the pressure values differed only by a mean of \SI{0.25}{\hecto\pascal} and a standard deviation of \SI{0.20}{\hecto\pascal}. The rain data was also expanded by using rain data from Örja since this data was gathered from September 1, 1986, to December 8, 1992. Thus, the rain data was used together with Hörby to categorize high-pressure blocking events over a large period of time. The Örja station was located \SI{14}{\km} from Vavihill and \SI{32}{\km} from Malmö.


\subsection{The meteorological measuring devices}
The wind data from both Hörby and Helsingborg used the high-performance wind sensor Vaisala WAA15A for the wind speed and Vaisala WAV15A for the wind direction. These instruments were serviced and calibrated every year or every other year, and had been in use since 1995. The WAA15A anemometer measured wind speed with an accuracy of \SI{0.17}{\m\per\s}, and the WAV15A wind vane measured the wind direction with an accuracy better than \ang{3}\cite{vaisalaWindSetWA152021}. The WAA15A anemometer works by a rotating chopper disc that interrupts an infrared beam, resulting in a laser pulse proportional to the wind speed. The WAV15A wind vane uses a counterbalanced vane with an optical disc. When the vane turns, infrared LEDs detect the change in angle with the disc and phototransistors, resulting in a precise measurement of the wind angle. For rain monitoring, the Geonor T200 device had been in use for all stations since 1995. Like the wind monitor, this device had been serviced and calibrated every year or every other year. This device works by measuring precipitation with a vibrating wire sensor that detects weight changes from the water droplets\cite{geonorinc.T200BSeriesAll2019}. The device has a measurement accuracy better than \SI{0.1}{\mm}.

The barometer that has been in use for Helsingborg is a Vaisala PTB201A for the entire period, except for the periods from April 15, 2015, to April 17, 2025, and from September 19, 2004, to May 23, 2014, when a Vaisala PTB220 was used instead. Even then, the device has been serviced every year or every other year. The PTB201A digital barometer operates using a silicon capacitive absolute pressure sensor, providing stable and accurate pressure values \cite{vaisalaPTB200DIGITAL1993}. The sensor functions by means of a flexed diaphragm inside a capacitor that bends in response to air pressure, causing a change in the capacitor’s distance and thus a variation in the current. This device measures pressure in the range of \SI{600}{\hecto\pascal} to \SI{1100}{\hecto\pascal}, with an accuracy of \SI{0.3}{\hecto\pascal}. Errors in the device may arise due to environmental factors, such as exposure to condensing gases. The Vaisala PTB220 digital barometer operates in a similar manner but offers a wider measurement range of \SI{500}{\hecto\pascal} to \SI{1100}{\hecto\pascal}, with an improved accuracy of \SI{0.15}{\hecto\pascal} \cite{vaisalaPTB220SeriesDigital2001}.


How was wind and rain in Malmö measured? 

How was pressure in Ängelholm  and rain in Örja measured? 

what PM2.5 measurement devices was used in Vavihill/ Malmö? 