\newpage
\section{Method} 

\subsection{The Data Handling and Devices}
The meteorological data was downloaded from the SMHI’s website as CSV files. The data included hourly atmospheric pressure data, hourly rain data, and hourly wind data (speed and direction). Hourly aerosol data as \PM, measured over one-hour intervals, was also downloaded. Since the purpose of this project was to analyse aerosol concentrations in southern Sweden during high-pressure blocking events, one urban and one rural site were selected. The locations were chosen based on their classification as rural or urban and the length of time the stations had been in operation, where more data was considered better. 

\subsubsection{The Aerosol Data and Measurements}
The rural measuring station in the county of Skåne with the most aerosol data was Vavihill located in Svalöv, southern Sweden. This station was active from September 28, 1999, to November 15, 2017. However, due to missing data on some days, only \SI{57}{\%} of the period contained non-missing values. Additionally, between 2017 and 2018, the Vavihill station was relocated to nearby Hallahus, where it operated from May 10, 2018, to December 31, 2022, with \SI{93}{\%} of the period containing non-missing values. Combining these datasets resulted in a total of \SI{5371}{} days of hourly data. For an urban location in the county of Skåne, Malmö Rådhuset had the most data, with measurements recorded from June 3, 1999, to December 31, 2023. Here, \SI{90}{\%} of the recorded values were non-missing, resulting in \SI{8074}{} days of data. 

The measurement device used at Vavihill was the ambient particulate monitor TEOM 1400. This monitor continuously collects airborne particles less than \SI{2.5}{\micro\meter} onto a filter and measures their mass using an oscillating microbalance technique \cite{thermofisherscientificinc.TEOMSeries1400a2007}. The oscillating microbalance works by vibrating at a natural frequency, which changes as particles accumulate on the filter. Since this frequency shift is proportional to the mass of the particles, their total mass can be calculated. The precision of the monitor was $\pm\SI{1.5}{\micro\gram\per\meter\cubed}$. Thus, the monitor provides high precision and stable measurements. However, a key limitation is that it cannot distinguish between different types of particles, as it only measures total mass.

When the measuring station was moved to Hallahus, the measuring device was updated to the fine dust analysis system Palas FIDAS 200. This device works by an optical aerosol spectrometer, which samples particles from an isokinetic inlet through a polychromatic light-scattering channel where the scattering angles and intensities were measured. \cite{palasgmbhOperatingManualFidas}. This results in a high accuracy of $\pm\SI{0.1}{\micro\gram\per\meter\cubed}$, indicating an improvement from the previous device. 

The monitoring station in Malmö used several measuring devices over time, in conjunction with other equipment. Between the start of the monitoring and January 1, 2009, the TEOM 1400 monitor was used. From January 1, 2009, to December 31, 2015, the TEOM 1400, FDMS, and 8500 B or CB dryer were employed. Between January 1, 2016, and December 31, 2021, the TEOM 1405F and FDMS systems, along with the 8500 B, were in use. Finally, from January 1, 2022, the Palas FIDAS 200 monitor replaced the earlier systems. The FDMS (Filter Dynamics Measurement System) is a dynamic filter measurement system that enhances measurements by accounting for volatile and semi-volatile particles \cite{thermoscientific8500FDMSFilter2010}. The CB dryer and 8500 B were air dryers used to prevent moisture from entering the measurement devices, ensuring accurate data by avoiding potential interference caused by water vapour.

\subsubsection{The Meteorological Data and Measurements}
The choice of atmospheric pressure measurement station was Helsingborg, located \SI{25}{\km} from Vavihill and \SI{49}{\km} from Malmö. This location was chosen based on its proximity to both \PM measuring stations, the fact that neither Malmö nor Vavihill have pressure measurements from this period, and the fact that the station has been in use from August 2, 1995, with measurements taken every hour without any missing values. The period used for this station was from start until October 10, 2024. This station thus covers the entire period of the \PM data. Although the data was measured at a certain altitude, the data was provided as recalculated sea-level pressure.

The barometer that has been in use for Helsingborg is a Vaisala PTB201A for the entire period, except for the periods from April 15, 2015, to April 17, 2025, and from September 19, 2004, to May 23, 2014, when a Vaisala PTB220 was used instead. The device has been serviced every year or every other year. The PTB201A digital barometer operates using a silicon capacitive absolute pressure sensor, providing stable and accurate pressure values \cite{vaisalaPTB200DIGITAL1993}. The sensor functions by means of a flexed diaphragm inside a capacitor that bends in response to air pressure, causing a change in the capacitor’s distance and thus a variation in the current. This device measures pressure in the range of \SI{600}{\hecto\pascal} to \SI{1100}{\hecto\pascal}, with an accuracy of $\pm\SI{0.3}{\hecto\pascal}$. Errors in the device may arise due to environmental factors, such as exposure to condensing gases. The Vaisala PTB220 digital barometer operates in a similar manner but offers a wider measurement range of \SI{500}{\hecto\pascal} to \SI{1100}{\hecto\pascal}, with an improved accuracy of $\pm\SI{0.15}{\hecto\pascal}$ \cite{vaisalaPTB220SeriesDigital2001}.

The relevant rain and wind data were gathered from two different stations. For Vavihill, the station at Hörby, located \SI{35}{\km} away, was used, and for Malmö, a weather station just \SI{6}{\km} away was used. The weather station at Hörby was chosen instead of Helsingborg since neither Vavihill nor Hörby are located along the coast, whereas Helsingborg is. Coastal regions can experience a daily cycle of sea breezes and land breezes, which alter the wind direction. The wind and rain data from Hörby were measured from August 1, 1995, to October 1, 2020, with measurements taken every hour without any missing values. The Hörby station was temporarily relocated for a short period in 2021. The rain data from Malmö was measured from November 21, 1995, and the wind data from January 1, 1990, with both measurements ending on December 31, 2020. These stations did not lack any data.

The wind data from both Hörby and Helsingborg used the high-performance wind sensor Vaisala WAA15A for the wind speed and Vaisala WAV15A for the wind direction. These instruments were serviced and calibrated every year or every other year, and had been in use since 1995. The WAA15A anemometer measured wind speed with an accuracy of $\pm\SI{0.17}{\m\per\s}$, and the WAV15A wind vane measured the wind direction with an accuracy better than $\pm\ang{3}$ \cite{vaisalaWindSetWA152021}. The WAA15A anemometer works by a rotating chopper disc that interrupts an infrared beam, resulting in a laser pulse proportional to the wind speed. The WAV15A wind vane uses a counterbalanced vane with an optical disc. When the vane turns, infrared LEDs detect the change in angle with the disc and phototransistors, resulting in a precise measurement of the wind angle. For rain monitoring, the Geonor T200 device had been in use for all stations since 1995. Like the wind monitor, this device had been serviced and calibrated every year or every other year. This device works by measuring precipitation with a vibrating wire sensor that detects weight changes from the water droplets \cite{geonorinc.T200BSeriesAll2019}. The device has a measurement accuracy better than $\pm\SI{0.1}{\mm}$. 

The final task of this study was to evaluate if high-pressure blocking events had become increasingly more common. For this evaluation, atmospheric pressure data from Ängelholm airport was used instead of Helsingborg due to the pressure data from Ängelholm being active from January 5, 1946, meaning it has been in service for 49 years longer than that from Helsingborg. The data period used was from start until October 1, 2024. This station is located \SI{44}{\km} from Vavihill and \SI{76}{\km} from Malmö. However, the pressure values differed only by a mean of \SI{0.25}{\hecto\pascal} and a standard deviation of \SI{0.20}{\hecto\pascal} between December 1, 1995 to October 1. The rain data was also expanded by using daily rain data from Ängelholm since this data was gathered from January 18, 1947, to November 30, 2001. To obtain the maximum amount of the rain data the data from Ängelholm was used together with the nearby station of Tånga. This station is located \SI{12}{\km} away and had been in use since December 19, 1973. For this station the period used was from start until August 31, 2024 with daily rain data. The rain measurements in Tånga was measured manually by a beaker located on a field. Thus, errors in this measurement were higher than the digital measurement devices. 

\subsection{The Identification of High-Pressure Blocking Events}
To evaluate the occurrence of high-pressure blocking event for Vavihill or Malmö, the rain data and atmospheric air pressure were used. For a period to be defined as a high-pressure event, the atmospheric pressure had to be over \SI{1014}{\hecto\pascal}, and the rainfall had to be less than \SI{0.5}{\mm\per\hour}. These values were based on the fact that \SI{1014}{\hecto\pascal} was the mean atmospheric pressure from Helsingborg, and \SI{0.5}{\mm\per\hour} was chosen since this is considered as very light rain. The rain-limit was set to \SI{2}{\mm\per\day} in the case where daily precipitation was used instead. This value was chosen since it corresponds to a small amount of rainfall for a day, and almost no rainfall should be observed during high-pressure blocking events. The choice of \SI{0.5}{\mm\per\hour} for hourly data is motivated by the fact that continuous light rain over a 24-hour period is very rare. As a result, daily precipitation levels seldom reach \SI{12}{\mm\per\day}, and are more commonly around \SI{2}{\mm\per\day}. Since a high-pressure blocking event covers a large geographical area, and Vavihill and Malmö are located close to each other, a blocking event observed at one site should also be detectable at the other. To account for this, all identified high-pressure blocking events at one location were required to correspond to an event at the other location within a maximum time difference of \SI{5}{\hour} for the initial five days. For a high-pressure event to be considered a high-pressure blocking event, the criteria for a high-pressure event had to persist for at least \SI{120}{\hour} (5 days). This value was chosen since a 5-day limit is often considered when classifying high-pressure blocking events \cite{lupoAtmosphericBlockingEvents2020}. 

\subsection{The Data Analysis}
The evolution of \PM during the high-pressure blocking events were evaluated by calculating the mean and standard deviation of \PM for each hour of all the high-pressure blocking events, to evaluate the average progression of aerosols over time. This was done using the Python packages \texttt{NumPy} and \texttt{pandas}. Since the high-pressure blocking events varied in length, a minimum of eight events was required when calculating the mean and standard deviation on the \PM data. The mean PM value was then compared with the mean \PM value during periods without high-pressure blocking events, as well as with the EU annual mean limit for \PM. Due to the lack of \PM data during some high-pressure blocking events, a filter was applied, stating that a period needed \SI{85}{\%} \PM coverage in order to be analysed.

The data was sorted in different ways to explore how the \PM concentration depended on different parameters. Firstly, the data was sorted into one of four wind categories: northeast (310° to 70°), southeast (70° to 190°), west (190° to 310°), and no specific direction. This was done by categorizing the data if \SI{60}{\%} of the wind directional data fell into one of these categories, with zero wind speed being handled as a missing value.

Secondly, the data was sorted based on the season of the blocking. This was evaluated by taking the midpoint date of the blocking and categorizing the season by the month it occupied. December, January, and February were considered winter; March, April, and May were considered spring; June, July, and August were considered summer and September, October, and November were considered autumn. Lastly, the data was categorized based on the strength of the high-pressure blocking, where a weak high-pressure blocking event had a mean atmospheric pressure between \SI{1014}{\hecto\pascal} and \SI{1020}{\hecto\pascal}, an event had a mean atmospheric pressure between \SI{1020}{\hecto\pascal} and \SI{1025}{\hecto\pascal}, and an event had a mean atmospheric pressure over \SI{1025}{\hecto\pascal}.

The last task of this study was to evaluate whether high-pressure blocking events had become increasingly more common. This was evaluated in two different ways: by calculating the number of days under high-pressure blocking events per year, and the number and lengths of high-pressure blocking events per year. The number of days of blocking was also sorted by the season of the blocking to provide more insight into the nature of the high-pressure blocking events. 

\subsection{Statistical Evaluation: The Mann–Kendall Test and Sen's Slope}
To evaluate statistical significance of the trends found in the study, the statistical Mann-Kendall test was used. The Mann-Kendall test was applied to evaluate whether the \PM mean during high-pressure blocking events had increased, and if so, by how much. The Mann-Kendall test is a non-parametric statistical test used to calculate the monotonic trend, the significance of the result and the linear trend of a dataset. This test is commonly used in climate physics due to the challenges posed by many parameters and complex distributions. The test works by calculating the difference between each time step in the dataset. The output will be a p-value below 0.05 if the test provides a significant result, meaning that the trend is unlikely to be caused by randomness. Kendall's $\tau$-value is used to evaluate the monotonic increase of the dataset, where -1 indicates a total monotonic decrease, 1 indicates a total monotonic increase, and 0 indicates no monotonic trend. The $\tau$-value can be summarized by the formula 
\begin{equation}
    \tau = \frac{C - D}{C + D,}
    \label{eq:Kendalltau}
\end{equation}
where $C$ is the number of concordant pairs and $D$ is the number of discordant pairs. If the result yielded a $\tau$-value above 0.5, the result was labelled as a clear increase. 

Sen's slope is a method, often used together with the Mann-Kendall test, of performing linear regression on the data. If a monotonic increase is shown by the Mann-Kendall $\tau$, Sen's slope provides an estimate of the magnitude of that increase. The method differs from the least squares method because it uses the median to calculate the slope. This ensures that outliers, which are common in weather data, do not affect the results. Sen's slope is calculated as 
\begin{equation}
    S_{i} = \frac{C_{i+1} - C_i}{t_{i+1} - t_i} \quad \text{Sen's slope} = \text{median}(S_{i}),
    \label{eq:Senslope}
\end{equation}
where $i$ represents the indices, $C$ represents the concentration of \PM, and $t$ represents time. The program used for the Mann-Kendall test was the \texttt{pymannkendall} package in Python \cite{hussainmd.PyMannKendallPythonPackage2019}.
