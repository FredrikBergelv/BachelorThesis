\newpage
\section{Discussion}
\subsection{Why an Increase Occurs After 9-13 Days for Stronger High-Pressure Systems}
The result above showed that for all high-pressure blocking events, except weaker and medium strong events, an aerosol increase was seen after nine to 13 days compared to the initial aerosol values. This result provides an important result, which is that the aerosol concentration accumulates during high-pressure blocking events. An interesting observation is that this occurs after this specific time. 

\subsection{Particle Advection Versus Local Emissions}
One important concept to discuss is whether the accumulation of aerosol is due to local emissions or particle advection from other locations. To address this question, several observations can be made: Firstly, local emissions in urban Malmö should be much higher than those in Rural Vavihill. Secondly, if the increase is due to particle advection, the increase should depend strongly on the wind direction. 

From all the plots regarding \PM, one can observe that the increase in Malmö is generally stronger than in Vavihill. This suggests, as predicted, that some of the increase in aerosol is due to local emissions. One could argue that this increase might be due to other factors, such as the fact that Malmö is coastal, whereas Vavihill is not, or that Malmö is located near regions with higher emissions, such as central Europe. However, this is unlikely to be the case. Firstly, the fact that Malmö is coastal would not suggest a higher aerosol increase than in Vavihill. The coastal factor may influence the overall air quality in the location, but since we are observing local changes, this factor should be accounted for. The argument is that Malmö is more closely located to central Europe, especially in the southeastern direction. This factor may play a role in the stronger increase in Malmö than Vavihill, due to the fact that stronger high-pressure blocking events indicate slower air movement which would imply  that locations closer to the locations of high emission would be impacted stronger than locations further away. However, the length scale to central Europe compared to the length scale between Vavihill and Malmö makes this factor irrelevant. Furthermore, other large urban areas such as Copenhagen are not closer to Malmö than Vavihill. 

If we examine the wind dependence, we observe another interesting result. For wind direction from the northwest, there is no noticeable increase in aerosols, whereas we see a stronger increase when the wind comes from the west, and an even stronger increase from the southeastern direction. This suggests that some of the particle increase is due to particle advection. More importantly, we observe a notable difference between the southeastern and the northwestern directions. One could argue that this directional increase has more to do with the different types of high-pressure blocking events. However, the different types of high-pressure events such as Hfa, Sea, and HM correspond to different types of wind movement which thus means that this is the same as mentioned above. 

Another argument which indicates that there is a directional dependence is comparing the data from the non-specified wind direction with the other wind directions. The non-specified wind direction showed a monotone increase for both locations; however, this increase is not as strong as the increase observed for the western and southeastern directions. Furthermore, the northeast direction showed a smaller increase than the non-specified direction. The non-specified direction could be the increase corresponding to the local emission, whereas the directional plots show how the local particle concentration is affected by the particles arriving with the wind. This would suggest that the wind from the northeast contains cleaner air, whereas the wind from the west and especially the southeast is exhibiting a higher aerosol concentration. 

\subsection{Differences Between the Urban and Rural Location}
In the plots above one could observe that the aerosol accumulation was clearer in the case for the urban location in comparison to the rural. Several reasons could explain this behaviour. Firstly, emissions in the urban location  should be higher than in the case of the rural location. This is supported by the seasonal dependence where similar behaviour of the aerosol concentrations is seen for both locations expect during the winter, where a larger increase is seen in the urban location. Since aerosols emissions increase due to domestic heating during the winter would one expect an stronger increase in the urban location during the winter. 

Another reason might be the difference in the subsidence inversion layer for the different locations. Although the natural description of the two locations are similar, the skyline due to buildings is not. The urban location has many buildings which create friction onto the ground surface, whereas the same is not true for the rural location which is characterized by farmland. This makes the ground friction stronger in the urban location than in the rural location, which results in a clearer subsidence inversion layer in the urban location. One would thus obtain more vertical air mixture in the rural location, which would contribute to the lower levels of aerosols in this location. 

However, since the relative aerosol levels mainly differ during the winter, one could note that the main difference between the rural and the urban location is the local emissions. One could argue that the ground inversion during the winter is amplified due to longer nights, which would amplify the inversion layer and prevent further air mixing. However, since the ground inversion is similar for the two locations, one would not see a difference between the locations. So, although this may contribute to the higher levels of aerosols during the winter, it would not explain the difference between the urban and rural locations. One can thus conclude that the larger emissions of the urban location, and a slightly stronger subsidence inversion, explain the higher aerosol levels in the urban location.
