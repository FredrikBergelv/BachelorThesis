\newpage
\section{Discussion}
The result above showed that for all high-pressure blocking events, except for events with winds from the northeast, an aerosol increase was seen after eight to thirteen days compared to the initial aerosol values. This outcome provides an important result, which is that the aerosol concentration accumulates during high-pressure blocking events. An interesting observation is that this occurs after this specific time. 

One important concept to discuss is whether the accumulation of aerosol is due to local emissions or particle advection from other locations. To address this question, several observations can be made: Firstly, local emissions in urban Malmö should be much higher than those in rural Vavihill. Secondly, if the increase is due to particle advection, the increase should depend strongly on the wind direction. 

From all the plots regarding \PM, one can observe that the increase in Malmö is generally stronger than in Vavihill. This suggests, as predicted, that some of the increase in aerosol is due to local emissions. One could argue that this increase might be due to other factors, such as the fact that Malmö is coastal, whereas Vavihill is not, or that Malmö is located near regions with higher emissions, such as central Europe. However, this is unlikely to be the case. Firstly, the fact that Malmö is coastal would not suggest a higher aerosol increase than in Vavihill. The coastal factor may influence the overall air quality in the location, but since we are observing local changes, this factor should be accounted for. The argument that Malmö is more closely located to central Europe, especially in the southeastern direction may play a role in the stronger increase in Malmö than Vavihill. This is due to the fact that stronger high-pressure blocking events indicate more stagnant air which would imply that locations closer to the locations of high emission would be impacted more strongly than locations further away. However, the length scale to central Europe compared to the length scale between Vavihill and Malmö makes this factor negligible. Furthermore, other large urban areas such as Copenhagen are not substantially closer to Malmö than Vavihill. 

If we examine the wind dependence, we observe another interesting result. For wind directions from the northeast, there is no noticeable increase in aerosols in Malmö, whereas we see a stronger increase when the wind comes from the west, and an even stronger increase from the southeastern direction. This suggests that some of the particle increase is due to particle advection. One could argue that this directional increase has more to do with the different types of high-pressure blocking events. However, the main difference between different types of high-pressure blocking events is where they are centred, which mainly correspond to different wind directions. 

Another argument which indicates that there is a directional dependence is comparing the data from the non-specified wind direction with the other wind directions. The non-specified wind direction showed a monotone increase for both locations; however, this increase is not as strong as the increase observed for the southeastern direction for Malmö and for Vavihill (if one observes the initial eight days). Furthermore, the northeast direction in Malmö showed a smaller increase than the non-specified direction. This would suggest that the wind from the northeast contains cleaner air, whereas the wind from the southeast is exhibiting a higher aerosol concentration. 

In the plots above one could observe that the aerosol accumulation was stronger in the case for the urban location in comparison to the rural. Several reasons could explain this behaviour. Firstly, emissions in the urban location should be higher than in the case of the rural location. This is supported by the seasonal dependence where similar behaviour of the aerosol concentrations is seen for both locations except during the winter, where a larger increase is seen in the urban location. Since aerosols emissions increase due to road transportation with winter tyres and worse road surfaces one would expect a stronger increase in the urban location during the winter. 

However, the aerosol advection from nearby countries should also be increased during the winter. In southern Sweden district heating is the most common type of domestic heating, whereas countries to the southeast and east, such as Poland, more heat produced by coal combustion. The process of coal combustion for domestic heating produces considerable amounts of aerosols, which could be dispersed or advected to southern Sweden. However, the observation that aerosol concentrations are much higher in Malmö than in Vavihill during winter suggests that advection from nearby countries does not explain the entire picture.

Another reason might be the difference in the inversion layer for the different locations. Although the natural description of the two locations are similar, the skyline due to buildings is not. The urban location has many buildings which hold moisture badly, whereas the rural is characterised by nearby trees and hills which are better at holding moisture. This makes the ground more dependent on the ground radiation, which results in a clearer ground inversion layer in the urban location. One would thus obtain more vertical air mixture in the rural location, which would contribute to the lower levels of aerosols in this location. 

Since the aerosol levels mainly differ during the winter, one could argue that the main difference between the rural and the urban location is the local emissions. However, one could note that the ground inversion during the winter is amplified due to longer nights, which would amplify the inversion layer and prevent further air mixing (just like the ground inversion during the summer should be limited). Even though the inversion is mainly affected by the subsidence inversion, which is similar for the two locations, the ground inversion should still affect the total inversion. One could thus conclude that the total inversion is increased during the winter due to the ground inversion amplifying the total inversion in both locations. The slightly stronger ground inversion in the urban location together with the larger emissions of the urban location would give a higher aerosol concentration.
