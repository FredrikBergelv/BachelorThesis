\subsection{The physics behind high pressure systems}
Anticyclones are meteorological high-pressure systems in which air sinks toward the ground, creating high pressure \cite{spiridonovCyclonesAnticyclonesSpringerLink2020}. This occurs due to the convergence of air from all directions at high altitudes, which forces the air to move downward. The descending air undergoes adiabatic compression, resulting in an increase in the energy of air molecules, or, in other words, a higher temperature. This rise in energy inhibits cloud formation, as warmer air can hold less moisture. The absence of clouds allows solar radiation to significantly impact the temperature during an anticyclone, leading to warmer temperatures during the day time and cooler temperatures during the night. Consequently, this leads to a large temperature difference between day and night, with summer anticyclones associated with high temperatures and winter anticyclones with low temperatures. 

The Navier-Stokes equation
\begin{equation} \frac{\partial \vec{v}}{\partial t} + (\vec{v} \cdot \nabla)\vec{v} = \vec{g} - \frac{\nabla p}{\rho} - 2\vec{\Omega} \times \vec{v}, \label{eq:NavierStokes} \end{equation}
explains the movement of the air during an anticyclone. In the \autoref{eq:NavierStokes} the first term corresponds to the local acceleration of the air, the second term corresponds to the convective acceleration of the air, the third term is simply the gravitational acceleration, the fourth term describes the force from the pressure gradient and the last term describes the Coriolis force. Since we want to examine the stable solution of this equation, we can assume that the local acceleration is zero. Furthermore, the weather system is a large scale system with the size of hundreds of kilometres, the speed is in the tens of meters per second and the Coriolis term is in the order of \SI{0.5e-4}{\per\s}. Using this information one can observe that the size of the convection term is negligible in comparison to the Coriolis term. Lastly, one can assume the wind direction to be in the horizontal plane since vertical movement is a lot slower, making $\vec{v}=(v_x,v_y,0)$. The Coriolis term can be assumed to the vertical direction as $\vec{\Omega}=(0,0,\Omega_z)$. The equation is thus simplified in the case of large scale weather systems to
\begin{equation} 0 = - \frac{\nabla p}{\rho} - 2{\vec{\Omega}} \times \vec{v}, \label{eq:NavierStokes2} \end{equation}
where the gravity is neglected since it lies in the vertical plane. One thus obtains the solution of the velocity as:
\begin{equation} v_x=-\frac{1}{2\rho\Omega_z}\frac{\partial p}{\partial y}\text{ and }v_y=\frac{1}{2\rho\Omega_z}\frac{\partial p}{\partial x}. 
\end{equation}
Examining the vorticity, one obtains  
\begin{equation}
    (\nabla \times \vec{v})_z  = \frac{1}{2\Omega_z\rho}\nabla^2p,
\end{equation} 
where the definition of the anticyclone is that the pressure is strongest in the origin ($\nabla^2p<0$). We have thus shown that he rotation of the anticyclone is in the clockwise direction in the Northern Hemisphere. 

Under normal conditions the temperature of the air in the atmosphere decreases with the altitude, due to the ideal gas law. The cooling per altitude is called the environmental lapse rate and makes it possible for vertical wind movement to occur, when a slight imbalance is introduced to the system. However, during the night the air closer to the ground will lose heat due to the outgoing radiation from the Earth. This process is especially strong due to the absence of clouds during the anticyclone. This creates an inversion layer, where the air temperature increases with height for the initial height from the ground \cite{gregohareWeatherClimateClimate2005}. This daily cycle prohibits vertical air movement in the atmosphere.

During an anticyclone, adiabatic compression of the air occurs towards the ground. This process increases the temperature further closer to the ground. However, this downward movement of the air does not reach the ground due to the friction opposed by buildings, valleys, etc. Thus, the downward draught will spread out a few hundred meters above the ground and thus not mix with the air that lies closest to the ground. From the adiabatic compression, another inversion will be created, which makes the air closest to the ground cooler than the air a few hundred meters above ground. This process is called a subsidence inversion and will prevent air mixing between the ground level and the upper layers of the atmosphere \cite{gramschInfluenceSurfaceSubsidence2014}.

A high-pressure blocking period refers to a prolonged anticyclone characterized by higher surface pressure covering a large area \cite{lupoAtmosphericBlockingEvents2020}. Since the blocking system extends over a vast region, the pressure gradient remains small due to minimal fluctuations. As a result, winds tend to be calm to gentle breezes. A blocking period is typically defined as lasting between five and ten days, although no single definition exists. While the concept has been recognized in meteorology for over a century, the long-term consequences of blocking events are not yet fully understood. High-pressure blocking periods are more common in the Northern Hemisphere compared to the Southern Hemisphere. Research has indicated that the frequency of blocking periods has increased in recent years \cite{lupoAtmosphericBlockingEvents2020}. 

Recurring anticyclones can be classified into Hess and Brezowsky macrocirculation types, such as the Fennoscandian High (Hfa), the Southeast Anticyclone (Sea), and the Central European High (HM) \cite{bartholyEuropeanCycloneTrack2006}. These anticyclones are located at specific geographic points. Since anticyclones exhibit winds rotating clockwise around their center, the winds from Hfa, Sea, and HM tend to blow toward southern Sweden from the south and east. The transport of airborne pollutants, such as ozone, can occur via these winds \cite{oteroImpactAtmosphericBlocking2022}. Consequently, it can be hypothesized that other airborne aerosols, such as \PM, should also be transported through these wind patterns.

Moving anticyclones

\subsection{The origin of aerosols}
The concentration of aerosols are measured through the \PM, which to particulate matter with an aerodynamic diameter of \SI{2.5}{\micro\meter} or less. Although these aerosols can form naturally in the atmosphere, their primary sources include solid fuel combustion for domestic heating, industrial activities, and road transportation \cite{europeanenvironmentagencyEuropesAirQuality2024}. Significant molecular contributors to the \PM levels include sulfur dioxide (SO\textsubscript{2}) and soot (black carbon), both originating from the burning of fossil fuels. The European Union has set an annual mean limit for \PM  concentrations at \SI{25}{\micro\gram\per\cubic\meter}. This threshold has been exceeded in several countries, including Croatia, Bosnia and Herzegovina, Italy, Poland, North Macedonia, and Türkiye \cite{europeanenvironmentagencyEuropesAirQuality2024}. 

Studies have demonstrated a correlation between elevated \PM concentrations and an increased risk of respiratory, cardiovascular, and cerebrovascular diseases, as well as diabetes \cite{sharmaHealthEffectsAssociated2020}. A Danish study on on \SI{49564}{} individuals between 1993 to 2015 showed that for every \SI{5}{\micro\gram\per
\m\cubed} increase in aerosol concentrations the hazard ratio increased by 1.29 \cite{hvidtfeldtLongtermResidentialExposure2019}. Thus, for every \SI{5}{\micro\gram\per
\m\cubed} increase in \PM the risk of dying from cardiovascular diseases increased by 29\%. This number increased even more for people with health issues or of older age.  

\subsection{Aerosol concentrations during high-pressure blocking events}
During an high-pressure blocking event the environmental lapse inversion, especially the subsidence inversion, close to the ground prohibits vertical air mixing during the atmospheric layers closest to the ground. If aerosols are produced at the ground level during this high-pressure blocking event, this would imply that the aerosols would not disperse vertically, implying a higher concentration on the ground level. Thus, one would expect higher concentrations of \PM during high-pressure blocking events. Studies in China have shown that the vertical dispersion of aerosols during high-pressure blocking events are inhibited, increasing the concentration of \PM in cities \cite{caiImpactBlockingStructure2020}

Since aerosol emissions are particularly high in countries central European countries such as Poland, anticyclonic winds from Hfa, Sea, and HM are expected to increase \PM  concentrations in southern Sweden \cite{europeanenvironmentagencyEuropesAirQuality2024}. These aerosols would be transported to southern Sweden via southerly to easterly winds during the anticyclone. If this occurs during a high-pressure blocking event, the aerosols may accumulate over the region while continuously being advected by southerly and easterly winds. Whether the same occurs in southern Sweden is of interest for further study.