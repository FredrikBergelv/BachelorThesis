\section{Outlook}
Future work on this subject should use more advanced classification of different types of high-pressure blocking events by using data from multiple mereological stations. This would create a clearer view of the actual movement of the anticyclone, which could be used to simulate the advection of aerosols. Comparing this simulation with the values from this thesis would give more insight into the actual advection of the aerosols. To study local emissions, one could monitor the local emission inventories and compare this to the aerosol concentrations during high-pressure blocking events to see the effect that these play. To analyse the role of the inversion layer during the event, one could monitor vertical atmospheric data to see the effect of the subsidence inversion on the aerosol concentrations.

To summarize, the analysis of aerosol concentrations is important to monitor the health risks associated with different meteorological phenomena. This thesis has shown that not only domestic emissions play a role in air quality, but that advection from nearby countries also plays an crucial role. This shows the magnitude of international work to prohibit the large-scale emissions of aerosols. As the climate changes, so does the weather, and understanding how this affects us is of the utmost importance.

